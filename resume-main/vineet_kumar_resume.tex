%%%%%%%%%%%%%%%%%%%%%%%%%%%%%%%%%%%%%%%%%
% Medium Length Professional CV
% LaTeX Template
%
% This template has been downloaded from:
% http://www.LaTeXTemplates.com
%
% Original author:
% Trey Hunner (http://www.treyhunner.com/)
%
% Important note:
% This template requires the resume.cls file to be in the same directory as the
% .tex file. The resume.cls file provides the resume style used for structuring the
% document.
%
%%%%%%%%%%%%%%%%%%%%%%%%%%%%%%%%%%%%%%%%%

%----------------------------------------------------------------------------------------
%	PACKAGES AND OTHER DOCUMENT CONFIGURATIONS
%----------------------------------------------------------------------------------------
\newcommand{\matlab}{{\sc Matlab }}
\newcommand{\matlabx}{{\sc Matlab}}
\newcommand{\mixten}{{\sc Mi\xten }}
\newcommand{\mixtenx}{{\sc Mi\xten}}
\newcommand{\xten}{{\sc X10 }}
\newcommand{\xtenx}{{\sc X10}}
\newcommand{\xtenb}{{\textbf{\textsc{X10}}}}
\newcommand{\discode}{{\sc Discode}}
\documentclass{resume} % Use the custom resume.cls style
\usepackage[hyphens]{url}
\usepackage{hyperref}
\usepackage{setspace} 

\usepackage[left=0.5in,top=0.25in,right=0.5in,bottom=0.2in, footskip=0.0in]{geometry} % Document margins
\usepackage{marginnote}

\name{\textsc{Vineet} \textsc{Kumar}} % Your name
\begin{document}
\begin{hSubsection}{\textbf{Contact}: 514-970-9179, vineet.kumar@mail.mcgill.ca}
{%\url{http://vineetkumar.xyz}
}{917 Avenue de Melrose, Montreal, QC, Canada - H4A 2R3}
\end{hSubsection}



%----------------------------------------------------------------------------------------
%SUMMARY SECTION
%----------------------------------------------------------------------------------------
\smallskip \smallskip 
\begin{rSection}{} \smallskip \begin{lSubsection} 
\item \textbf{6 years of relevant experience} in software design and
development. $\vert$ \textbf{M.Sc Computer Science}, McGill University. 
\item Programming languages - \emph{Proficient:} Python,
  C++, Java
$\vert$ \emph{Prior experience:} \xtenx, \matlabx.
\end{lSubsection}
\end{rSection}

%------------------------------------------------------------------------------
%	WORK EXPERIENCE SECTION
%---------------------------------------------------------------------------
\begin{rSection}{Work Experience}

\begin{rSubsection}{INRO}{December 2014 - present}{Senior developer}{Montreal,
  QC, Canada}
\item Contribute to research and development of new features in INRO's two
	major products, Emme and CityPhi.
\item Built a new matrix calculator tool by writing a new parser and evaluator
	for Emme's matrix expression language.% (\textbf{Python} and \textbf{C++})
	(\href{http://bit.ly/1N5geA8}{\em{bit.ly/1N5geA8}})
\begin{lsubSubsection} 
\item Achieved \textbf{1.5x} to \textbf{30x} speedup depending on expression
	details and hardware resources.
\item Built a memory management system to efficiently handle computations on
	matrices over 1GB in size.
\end{lsubSubsection}
\item Data analytics for GTFS and travel smart-card data: Developed techniques and tools
  to visually analyze public transport smart card data.% (in \textbf{Python}).
  These tools can be used to analyze and query things like loads, delays, and stop activities.
\item Maintain and develop new Python APIs for Emme's Fortran backend.  
\begin{lsubSubsection}
\item Design input specifications for the new APIs that are also used to
	interface with the GUI frontend.
\item Validate and process inputs to generate Emme macros to interface with the
	Fortran backend.
\end{lsubSubsection}
\item Design and Develop CityPhi API for importing geo and transport data from
	various data formats like Shapefiles, OSM and GTFS. Also contribute to
	CityPhi's core data backend.% (in \textbf{Python} and \textbf{C++}).
\end{rSubsection}

\begin{rSubsection}{ISENCORE Technologies}{September 2013 - December 2014}{CTO
  and co-founder}{Montreal, QC, Canada}
\item Implemented (\textbf{in C}) the 3D object discretization module for \textbf{Quirdity},
  ISENCORE's 3D simulation engine.
\item Won \textbf{first prize} in the \textbf{Mcgill Dobson cup} 2014 startup competition. 
\item Delivered the \textbf{winning pitch} to get selected as \textbf{one of the 20 startups
  worldwide} to present at SLUSH 2014.
\end{rSubsection}

\begin{rSubsection}{McGill University}{January 2012 - April 2014}{Research and
Teaching}{Montreal, QC, Canada} 
\item \textbf{Research Assistant, Sable Lab} - My research included program
analysis and static compilation of dynamic languages.

\begin{lsubSubsection}
\item Designed and developed (\textbf{in Java}) \mixtenx: a \matlab to
\xten  compiler for high-performance, under \textbf{Prof.
Laurie Hendren's} supervision and with direct inputs from the \textbf{\xten
design team at the IBM T.J. Watson research
center}.(\href{http://bit.ly/1sZ8aqJ}{\em{bit.ly/1sZ8aqJ}})  
 \item Achieved \textbf{7 times (mean) faster} performance compared to the
 standard \matlab implementation.
  \item Discovered \textbf{2 bugs} and a \textbf{severe performance bottleneck}
in the \xten compiler.  
\end{lsubSubsection}
\item \textbf{Teaching Assistant} - Program Analysis and Transformations,
	Compiler Design, and Introduction to Computer Systems.
\end{rSubsection}

\begin{rSubsection}{Infosys Technologies Ltd.}{September 2008 - August
2011}{Systems Engineer}{Pune, India}
\item \textbf{Led} a team of 4 for \textbf{deployment performance management}
	for AT\&T's online and mobility frontend and backend applications.
\begin{lsubSubsection}
\item My team's job was to design and develop performance test scripts, analyze
	results, and troubleshoot performance issues.  
\item {Worked on 8 projects} and they all exceeded performance SLA under peak
	loads.
\end{lsubSubsection}
\end{rSubsection}

\begin{rSubsection}{Sun Microsystems}{January 2007 - May 2008}{Intern - Student
Tech Lead, APAC region/Campus Ambassador}{Bangalore, India}
\item Promoted from being one of the \textbf{only 27 Campus Ambassador across
	India} to one of the \textbf{only 5 Tech Leads worldwide}.  
\begin{lsubSubsection}
\item \textbf{Taught} a course on OpenSolaris at the university. Conducted
	webinars and developed tutorials for ambassadors worldwide.
\end{lsubSubsection}
\end{rSubsection}
\end{rSection}

%------------------------------------------------------------------------------
%	PUBLICATIONS SECTION
%------------------------------------------------------------------------------

\begin{rSection}{Publications}
\smallskip
\begin{lSubsection}


\item Vineet Kumar and Laurie Hendren. \mixten: Compiling \matlab to \xten for
	High Performance. In Proceedings of the 2014 ACM International
	Conference on \textbf{Object Oriented Programming Systems Languages \&
	Applications (OOPSLA
	`14)}.(\href{http://bit.ly/1sft0PU}{\em{bit.ly/1sft0PU}})

%\item \emph{Talk:} Vineet Kumar and Laurie Hendren. \mixten: Compiling \matlab
%for high performance computing via \xten. 12\textsuperscript{th}
%\textbf{Compiler-Driven Performance Workshop} at \textbf{CASCON
%2013}.(\href{http://webdocs.cs.ualberta.ca/~amaral/cascon/CDP13/#VinetKumar}{\em{bit.ly/1hXms8N}}) 
%
\item Vineet Kumar and Laurie Hendren. First steps to compiling \matlab to
	\xten. In Proceedings of the 2013 ACM SIGPLAN X10 Workshop, \textbf{X10
	`13} co-located with \textbf{PLDI
	2013}.(\href{http://www.sable.mcgill.ca/mclab/mix10/paper.pdf}{\em{bit.ly/18owBUI}})
\end{lSubsection}
\end{rSection}

%-----------------------------------------------------------------------------
%	EDUCATION SECTION
%-----------------------------------------------------------------------------
 
\begin{rSection}{Education}

\begin{rSubsection}{McGill University}{April 2014}{M.Sc. in
Computer Science (CGPA: 3.56/4.00)}{Montreal, QC, Canada} 
\item Master's thesis reviewed as \textbf{``Excellent''} by the external
reviewer.
\end{rSubsection}

\begin{rSubsection}{SASTRA University}{June 2008}{B.Tech. in Computer Science
\& Engineering (CGPA: 8.93/10.00)}{Thanjavur, India} 
\item Won the {Dean's list scholarship} for being among the \textbf{top
10\%} students in the University. 
%\item \textbf{Co-founded and led} GLOSS(GNU Linux \& Open Source at SASTRA)
%club of the University.  
\end{rSubsection}

\end{rSection}

%------------------------------------------------------------------------------
%PROJECTS SECTION TODO discode, distributed systems, networking
%------------------------------------------------------------------------------
\begin{rSection}{Selected other projects}\smallskip \begin{lSubsection}
%\item Performance
%Analysis and comparison of ZeroMQ and TCP (COMP 535 Computer networks, team of
%three).(\href{http://bit.ly/1m0dcPE}{\em{bit.ly/1m0dcPE}})  
%\begin{lsubSubsection}
%\item Implemented a ZeroMQ based P2P chat system and compared its throughput
%  and latency to a TCP based P2P chat system.
%\end{lsubSubsection}
%
\item Analysis to identify complex numerical values for \matlab programs (COMP
621 Program analysis,
individual).(\href{https://github.com/Sable/mclab/tree/master/languages/Natlab/src/natlab/tame/valueanalysis/components/isComplex}{\em{bit.ly/15SYKmC}}) 
\begin{lsubSubsection}
\item Developed a language to express information propagation through library
function calls.(\href{http://bit.ly/1ezq93q}{\em{bit.ly/1ezq93q}})
%\item Accurate results for all the 20 benchmarks used by the 
%McLab project.   
\end{lsubSubsection}

\item FreeMeLegal: An Open source license recommendation
engine (COMP 762 Recommender systems,
individual).(\href{http://bit.ly/1m030GV}{\em{bit.ly/1m030GV}}) 
%\begin{lsubSubsection}
%\item Recommendations based on similarities with the top projects on
%Sourceforge.net.
%\item Implemented a crawler (in {PHP}) to collect data for top projects 
%on Sourceforge.net.
%\end{lsubSubsection}
\end{lSubsection}
\end{rSection}
%\newpage

%-----------------------------------------------------------------------------
\end{document}
