%%%%%%%%%%%%%%%%%%%%%%%%%%%%%%%%%%%%%%%%%
% Medium Length Professional CV
% LaTeX Template
%
% This template has been downloaded from:
% http://www.LaTeXTemplates.com
%
% Original author:
% Trey Hunner (http://www.treyhunner.com/)
%
% Important note:
% This template requires the resume.cls file to be in the same directory as the
% .tex file. The resume.cls file provides the resume style used for structuring the
% document.
%
%%%%%%%%%%%%%%%%%%%%%%%%%%%%%%%%%%%%%%%%%

%------------------------------------------------------------------------------
%	PACKAGES AND OTHER DOCUMENT CONFIGURATIONS
%------------------------------------------------------------------------------
\newcommand{\matlab}{{\sc Matlab }}
\newcommand{\matlabx}{{\sc Matlab}}
\newcommand{\mixten}{{\sc Mi\xten }}
\newcommand{\mixtenx}{{\sc Mi\xten}}
\newcommand{\xten}{{\sc X10 }}
\newcommand{\xtenx}{{\sc X10}}
\newcommand{\xtenb}{{\textbf{\textsc{X10}}}}
\newcommand{\discode}{{\sc Discode}}
\documentclass{resume} % Use the custom resume.cls style
\usepackage[hyphens]{url}
\usepackage{hyperref}
\usepackage{setspace} 

\usepackage[left=0.5in,top=0.25in,right=0.5in,bottom=0.2in, footskip=0.0in]{geometry} % Document margins
\usepackage{marginnote}

\name{\textsc{Vineet} \textsc{Kumar}} % Your name
\begin{document}
\begin{hSubsection}
{
    \href{https://github.com/vineet7kumar}{github.com/vineet7kumar},
    \href{https://linkedin.com/in/vineet7kumar}{linkedin.com/in/vineet7kumar}
}
{
    \textbf{vineet.kumar@mail.mcgill.ca, 514-970-9179}
}
{%917 Avenue de Melrose, Montreal, QC, Canada - H4A 2R3}
}
\end{hSubsection}



%------------------------------------------------------------------------------
%SUMMARY SECTION
%------------------------------------------------------------------------------
\smallskip \smallskip 
\begin{rSection}{} \smallskip 
\begin{lSubsection} 
\item \textbf{6+ years of relevant experience} in software design and
        development. $\vert$ \textbf{M.Sc Computer Science}, McGill University. 
\item Programming languages - \emph{Proficient:} Python, C++, Java
        $\vert$ \emph{Prior experience:} \xtenx, \matlabx.
\item Skills and interests: Compilers, data analytics,
        distributed computing, web backend systems.
\end{lSubsection}
\end{rSection}

%------------------------------------------------------------------------------
%	WORK EXPERIENCE SECTION
%------------------------------------------------------------------------------
\begin{rSection}{Work Experience}

\begin{rSubsection}{INRO}{December 2014 - present}{Senior developer}{Montreal,
        QC, Canada}
\item Contribute to research and development of \textbf{Emme} and
        \textbf{CityPhi}. Emme is a travel demand modelling system for
        transportation forecasting. CityPhi is a visual analytics software for
        large-scale spatial and mobility data.
%\item Built a new matrix calculator tool by writing a new parser and evaluator
	%for Emme's matrix expression language.% (\textbf{Python} and \textbf{C++})
	%(\href{http://bit.ly/1N5geA8}{\em{bit.ly/1N5geA8}})
%\begin{lsubSubsection} 
%\item Achieved \textbf{1.5x} to \textbf{30x} speedup depending on expression
	%details and hardware resources.
%\item Built a memory management system to efficiently handle computations on
        %matrices over 1GB in size. (Python and C++) 
%\end{lsubSubsection}
%\item Built a new matrix calculator tool for Emme's matrix expression language
        %by writing a new parser, evaluator and memory management system. It is
        %upto 30x faster than the old calculator and efficiently handles
        %computations on matrices over 1GB in size. 
        
\item Built a new matrix calculator for Emme, \textbf{upto 30x faster and
        efficient on large matrices (over 1GB)}. Wrote a new expression parser,
        evaluator and memory management system. (Python and C++)
\item \textbf{Data analytics for GTFS and travel smart-card data}: Developed
        techniques and tools to visually analyze and query public transit smart card
        data. These tools are used by clients to analyze and query things like
        loads, delays, and stop activities. (Python)
\item Designed and developed \textbf{new Python API modules} for Emme's Fortran
        backend using Emme's macro language as interface. 
%\item Work on optimizing and extending CityPhi's core data backend. Also
        %Designed and Developed CityPhi API for importing geo and transport data
        %from various data formats like Shapefiles, OSM, and GTFS.
\item Designed and built Cityphi's \textbf{data import backend and API} to
        import geo and transport data from various formats like Shapefiles, OSM
        and GTFS.  Also work on optimizing and extending the core data backend.
        (C++ and Python)
\end{rSubsection}

\begin{rSubsection}{ISENCORE Technologies}{September 2013 - December 2014}{CTO
        and co-founder}{Montreal, QC, Canada}
\item Developed the \textbf{3D object discretization} module for
        \textbf{Quirdity}, ISENCORE's 3D simulation engine. It generates a
        voxel tree and the associated data for a 3D model.
        (C++)(\href{http://bit.ly/discretizer}{bit.ly/discretizer})
\item Won \textbf{first prize} in the \textbf{Mcgill Dobson cup} 2014 startup
        competition. 
\item Delivered the \textbf{winning pitch} to get selected as \textbf{one of
        the 20 startups worldwide} to present at SLUSH 2014.
\end{rSubsection}

\begin{rSubsection}{McGill University - Sable Compilers Research Lab}{January
        2012 - April 2014}{Research and Teaching}{Montreal, QC, Canada} 
\item \textbf{Research Assistant, Sable Lab} - My research included program
        analysis and static compilation of dynamic languages.

\begin{lsubSubsection}
\item Designed and developed \textbf{\mixtenx: a \matlab to \xten  compiler for
        high-performance}, under \textbf{Prof. Laurie Hendren's} supervision
        and with direct inputs from the \textbf{\xten design team at the IBM
        T.J. Watson research center}.
        (Java)(\href{http://bit.ly/getmix10}{\em{bit.ly/getmix10}})  
 \item Achieved \textbf{7 times (mean) faster} performance compared to the
 standard \matlab implementation.
  \item Discovered a \textbf{severe performance bottleneck}
in the \xten compiler.  
\end{lsubSubsection}
\item \textbf{Teaching Assistant} - Program Analysis and Transformations,
	Compiler Design, and Introduction to Computer Systems.
\end{rSubsection}

\begin{rSubsection}{Infosys Technologies Ltd.}{September 2008 - August
        2011}{Systems Engineer}{Pune, India}
\item \textbf{Led} a team of 4 for \textbf{deployment performance management}
	for AT\&T's online and mobility frontend and backend applications.
\begin{lsubSubsection}
\item My team's job was to design and develop performance test scripts, analyze
	results, and troubleshoot performance issues.  
\item {Worked on 8 projects} and they all exceeded performance SLA under peak
	loads.
\end{lsubSubsection}
\end{rSubsection}

\begin{rSubsection}{Sun Microsystems}{January 2007 - May 2008}{Intern - Student
        Tech Lead, APAC region/Campus Ambassador}{Bangalore, India}
\item Promoted from being one of the \textbf{only 27 Campus Ambassador across
	India} to one of the \textbf{only 5 Tech Leads worldwide}.  
\begin{lsubSubsection}
\item \textbf{Taught} a course on OpenSolaris at the university. Conducted
	webinars and developed tutorials for ambassadors worldwide.
\end{lsubSubsection}
\end{rSubsection}
\end{rSection}

%------------------------------------------------------------------------------
%	PUBLICATIONS SECTION
%------------------------------------------------------------------------------

\begin{rSection}{Publications}
\smallskip
\begin{lSubsection}


\item Vineet Kumar and Laurie Hendren. \mixten: Compiling \matlab to \xten for
	High Performance. In Proceedings of the 2014 ACM International
	Conference on \textbf{Object Oriented Programming Systems Languages \&
	Applications (OOPSLA
	`14)}.(\href{http://bit.ly/1papr1}{\em{bit.ly/1papr1}})

%\item \emph{Talk:} Vineet Kumar and Laurie Hendren. \mixten: Compiling \matlab
%for high performance computing via \xten. 12\textsuperscript{th}
%\textbf{Compiler-Driven Performance Workshop} at \textbf{CASCON
%2013}.(\href{http://webdocs.cs.ualberta.ca/~amaral/cascon/CDP13/#VinetKumar}{\em{bit.ly/1hXms8N}}) 
%
\item Vineet Kumar and Laurie Hendren. First steps to compiling \matlab to
	\xten. In Proceedings of the 2013 ACM SIGPLAN X10 Workshop, \textbf{X10
	`13} co-located with \textbf{PLDI
	2013}.(\href{http://bit.ly/2papr2}{\em{bit.ly/2papr2}})
\end{lSubsection}
\end{rSection}

%-----------------------------------------------------------------------------
%	EDUCATION SECTION
%-----------------------------------------------------------------------------
 
\begin{rSection}{Education}

\begin{rSubsection}{McGill University}{April 2014}{M.Sc. in
Computer Science}{Montreal, QC, Canada} 
\item Master's thesis reviewed as \textbf{``Excellent''} by the external
reviewer.
\end{rSubsection}

\begin{rSubsection}{SASTRA University}{June 2008}{B.Tech. in Computer Science
\& Engineering}{Thanjavur, India} 
\item Won the {Dean's list scholarship} for being among the \textbf{top
10\%} students in the University. 
%\item \textbf{Co-founded and led} GLOSS(GNU Linux \& Open Source at SASTRA)
%club of the University.  
\end{rSubsection}

\end{rSection}

%------------------------------------------------------------------------------
%PROJECTS SECTION TODO discode, distributed systems, networking
%------------------------------------------------------------------------------
\begin{rSection}{other projects}\smallskip \begin{lSubsection}
%\item Performance
%Analysis and comparison of ZeroMQ and TCP (COMP 535 Computer networks, team of
%three).(\href{http://bit.ly/1m0dcPE}{\em{bit.ly/1m0dcPE}})  
%\begin{lsubSubsection}
%\item Implemented a ZeroMQ based P2P chat system and compared its throughput
%  and latency to a TCP based P2P chat system.
%\end{lsubSubsection}
%
\item Analysis to identify complex numerical values for \matlab
        programs.(\href{http://bit.ly/iscomplex}{\em{bit.ly/iscomplex}}) 
\begin{lsubSubsection}
\item Developed a language to express information propagation through library
function calls.%(\href{http://bit.ly/1ezq93q}{\em{bit.ly/1ezq93q}})
\end{lsubSubsection}

\item FreeMeLegal: An Open source license recommendation
engine.(\href{http://bit.ly/freemelegal}{\em{bit.ly/freemelegal}}) 
\end{lSubsection}
\end{rSection}
%\newpage

%-----------------------------------------------------------------------------
\end{document}
