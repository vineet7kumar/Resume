%%%%%%%%%%%%%%%%%%%%%%%%%%%%%%%%%%%%%%%%%
% Medium Length Professional CV
% LaTeX Template
%
% This template has been downloaded from:
% http://www.LaTeXTemplates.com
%
% Original author:
% Trey Hunner (http://www.treyhunner.com/)
%
% Important note:
% This template requires the resume.cls file to be in the same directory as the
% .tex file. The resume.cls file provides the resume style used for structuring the
% document.
%
%%%%%%%%%%%%%%%%%%%%%%%%%%%%%%%%%%%%%%%%%

%----------------------------------------------------------------------------------------
%	PACKAGES AND OTHER DOCUMENT CONFIGURATIONS
%----------------------------------------------------------------------------------------
\newcommand{\matlab}{{\sc Matlab }}
\newcommand{\mixten}{{\sc Mi\xten }}
\newcommand{\xten}{{\sc X10 }}
\newcommand{\discode}{{\sc Discode}}
\documentclass{resume} % Use the custom resume.cls style
\usepackage[hyphens]{url}
\usepackage{hyperref}
%\PassOptionsToPackage{hyphens}{url}\usepackage{hyperref}
\usepackage{setspace} 

%\usepackage[left=0.75in,top=0.6in,right=0.75in,bottom=0.6in]{geometry} % Document margins
\usepackage[left=0.5in,top=0.2in,right=0.5in,bottom=0.2in, footskip=0.0in]{geometry} % Document margins
%\usepackage[left=0.2in,top=0.2in,right=1.8in, bottom=0.4in, marginparwidth=1.45in, marginparsep=0.1in, footskip=0.1in]{geometry} % Document margins
\usepackage{marginnote}

\name{Vineet Kumar} % Your name
\begin{document}
\begin{hSubsection}{514-970-9179, vineet.kumar@mail.mcgill.ca}
{%\url{http://vineetkumar.net}
}{3498 Rue Hutchison, Apt\#501, Montreal, QC, Canada - H2X 2G4}
\end{hSubsection}



%----------------------------------------------------------------------------------------
%	SUMMARY SECTION
%----------------------------------------------------------------------------------------
%\marginnote{{\textmd{\textsl{\scriptsize{\hrule Associated with some of the best people, 
%schools and companies in Computer Science      \smallskip\\}}}}}[0.541cm]
\begin{rSection}{Summary}
\smallskip
\begin{lSubsection}
\item I am seeking a \textbf{full-time} software development position
	\textbf{starting February 2014}.
\item \textbf{M.Sc Computer Science}, McGill University $\vert$
	\textbf{3+ years} of relevant \textbf{industry experience} in software
	 development and programming (C and Java) $\vert$ Specialization in
	 compiler design, program analysis and program transformations.
  \item Interests - Compilers, Computer networks, Recommender systems, Social networks, E-commerce, High-performance computing. 
 \item Skill set
 \begin{lsubSubsection}
 	\item Programming languages - Expert: C, Java, \xten, \matlab $\vert$ Proficient: PHP $\vert$ Prior experience: Python, Javascript.
 	\item Software development - GNU/Linux, Windows 7 $\vert$ MySQL $\vert$ Eclipse $\vert$ GCC, OpenJDK 7, JUnit, GDB 
		$\vert$ Git, SVN $\vert$ Vim, \LaTeX. 
 	\item Web development - Apache HTTP server, HTML, JSON, XML, RSS, APIs: Facebook, Twitter, Paypal, SEO, Amazon EC2.
 	\item Compiler design tools - Flex, Bison, Beaver, JastAdd.
 \end{lsubSubsection}
\end{lSubsection}
\end{rSection}


%----------------------------------------------------------------------------------------
%	EDUCATION SECTION
%----------------------------------------------------------------------------------------
%\marginnote{{\textmd{\textsl{\scriptsize{\hrule Graduate courses - Compiler Design, 
%Program Analysis and Transformations, Advanced Techniques for Compiling Dynamic Languages, Distributed Systems, Computer Networks, 
%Recommendation Systems for Software Engineering         \smallskip\\
%Active participation in extra-curricular activities     \smallskip\\ 
%}}}}}[0.541cm]
 \begin{rSection}{Education}

\begin{rSubsection}{McGill University}{December 2013 (expected)}{M.Sc. in Computer Science (CGPA: 3.56/4.00)}{Montreal, QC, Canada}
\item Selected for presentation at 12th \textbf{Compiler-Driven Performance Workshop} at \textbf{CASCON 2013}.
\end{rSubsection}
\begin{rSubsection}{SASTRA University}{June 2008}{B.Tech. in Computer Science \& Engineering (CGPA: 8.93/10.00)}{Thanjavur, India}
\item Won the  \textbf{dean's list scholarship for being in the top 10\% }of the University. 
\item \textbf{Co-founded and led} GLOSS(GNU Linux \& Open Source at SASTRA) club of the University.
\item \textbf{Executive member and member of editorial team} of Student Association of School of Computer Science. 
\end{rSubsection}

\end{rSection}

%----------------------------------------------------------------------------------------
%	PUBLICATIONS SECTION
%----------------------------------------------------------------------------------------

\begin{rSection}{Publications}
\smallskip
\begin{lSubsection}
\item Vineet Kumar and Laurie Hendren. First steps to compiling \matlab to \xten. In Proceedings of the 2013 ACM SIGPLAN X10 Workshop, \textbf{X10 `13} co-located with \textbf{PLDI 2013}.
\end{lSubsection}
\end{rSection}


%----------------------------------------------------------------------------------------
%	WORK EXPERIENCE SECTION
%----------------------------------------------------------------------------------------
%\marginnote{{\textmd{\textsl{\scriptsize{\hrule  
%Reserach, Teaching, Compilers, Program analysis, Program transformation and optimization techniques, Program language design,Code generation, 
%Java, MATLAB, JastAdd, Flex, Bison, Beaver, \matlab, \xten      \smallskip\\}}}}}[0.541cm]%bad hack but works
\begin{rSection}{Experience}
\begin{rSubsection}{McGill University}{January 2012 - Present}{Research and Teaching}{Montreal, QC, Canada}
\item \textbf{Research Assistant, Sable Lab} - My research is about static compilation and program analysis for dynamic languages.
\begin{lsubSubsection}
	\item Designed and developed a \textbf{\matlab compiler for high-performance computing} via \xten language under supervision from \textbf{Prof. Laurie Hendren} and with direct inputs from \textbf{\xten design team at IBM T.J. Watson Research Center}.
	\item Achieved \textbf{20-40\% performance improvements} over commercial \matlab implementation (for sequential code).
	\item Discovered a critical shortcoming in the \xten Java backend.
\end{lsubSubsection}
\item \textbf{Teaching Assistant} - Program Analysis and Transformations,
	Compiler Design, Introduction to Computer Systems.
\end{rSubsection}

%------------------------------------------------
%\marginnote{{\textmd{\textsl{\scriptsize{\hrule 
%Software development, Software engineering, Management, SEO, C, HP LoadRunner        \smallskip\\}}}}}[-0cm]
\begin{rSubsection}{Infosys Technologies Ltd.}{September 2008 - August 2011}{Senior Systems Engineer}{Pune, India}
\item I \textbf{led} a team of 4 for \textbf{deployment performance management} for AT\&T's online and mobility frontend and backend applications.
\begin{lsubSubsection}
\item My team's job was design and development of performance tests, analysis of results and troubleshooting performance issues.
\item \textbf{Worked on 8 projects} and they all exceeded performance SLA under peak loads.
\end{lsubSubsection}
\item Part of the development team for \textbf{SEO recommendation system} used by the \textbf{Infosys global marketing team}.
\end{rSubsection}

%------------------------------------------------
%\marginnote{\textmd{\textsl{\scriptsize{\hrule e-commerce, startup, social network APIs, PHP, Apache, MySQL          \smallskip\\}}}}[0.0cm]
\begin{rSubsection}{MySmartPrice}{October 2010 - July 2011}{Developer (part-time)}{Pune, India}
\item MySmartPrice is an Indian \textbf{e-commerce startup} that provides a price comparison engine for online stores. 
\begin{lsubSubsection}
\item I helped the co-founders with designing the core crawler algorithm. 
\item I Wrote the initial \textbf{social network interfaces} for facebook and twitter.
\end{lsubSubsection}
\end{rSubsection}

%\marginnote{\textmd{\textsl{\scriptsize{\hrule Teaching, Evangelism, Open Source culture, OpenSolaris         \smallskip\\}}}}[0.0cm]
\begin{rSubsection}{Sun Microsystems}{January 2007 - May 2008}{Intern - Student Tech Lead, APAC region/Campus Ambassador}{Bangalore, India}
%\item Student Tech Lead/Campus Ambassador, APAC region \\
\item I was promoted from being one of \textbf{only 27 Campus Ambassador across India} to one of \textbf{only 5 Tech Leads worldwide}.
\begin{lsubSubsection}
\item My job was to conducted webinars and develop tutorials for Campus Ambassadors worldwide.
%\item Campus Ambassador, India \\
%\item As a Campus Ambassador I was one of \textbf{only 27 Campus Ambassadors across India}.
\item \textbf{Taught} a certificate course on \textbf{OpenSolaris}.
\end{lsubSubsection}
\end{rSubsection}
\end{rSection}
%\newpage


%----------------------------------------------------------------------------------------
%	PROJECTS SECTION
%TODO discode, distributed systems, networking
%----------------------------------------------------------------------------------------
%\marginnote{\textmd{\textsl{\scriptsize{\hrule Data flow analysis, Open Source licenses, Compilers, Twitter API, OAuth, RSS, PHP, C, Javascript      \smallskip\\}}}}[0.541cm]
\begin{rSection}{Projects (Course and Personal)}
\smallskip
\begin{lSubsection}
	\item Discode: A decompiler for C on OpenSolaris for intel x86 (B.Tech project).
			(\href{http://vineetkumar.net/discode/}{\em{url}})
    \item Analysis to identify if a value is complex or real for \matlab programs (Course project for Program analysis).
    (\href{https://github.com/Sable/mclab/tree/master/languages/Natlab/src/natlab/tame/valueanalysis/components/isComplex}{\em{url}}) 
    \item FreeMeLegal: An Open source license recommendation engine (Course project for Recommender systems). 
    (\href{http://vineetkumar.net/freeMeLegal/free-me-legal_final_report.pdf}{\em{url}})
    \item WIG (a subset of \verb+<+bigwig\verb+>+ language) to PHP compiler (Course project for Compiler design in a team of two).
    \item Performance Analysis and comparison of ZeroMQ and TCP (Course project  for Computer networks in a team of three).
    \item Tweetinawhile: A web-based tweet scheduler.
    \item DotGyan: A server side browser independent search enhancement tool for web pages with textual content.
    (\href{http://vineetkumar.net/dotgyan/}{\em{url}})
%    \item Damruu: A RSS feed reader and chronologically based feed merger.
\end{lSubsection}
\end{rSection}
%\newpage

%----------------------------------------------------------------------------------------
\end{document}
